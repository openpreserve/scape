\section{Installation}

This section describes how to build Matchbox from the source distribution. There might be binary distributions provided...TODO


\subsection{Requirements}

\subsubsection{Open Computer Vision Library (OpenCV)}

The Open Source Computer Vision (OpenCV) libarary \cite{opencv_library} is a library of programming functions for real time computer vision.
It is released under a Berkeley Software Distribution (BSD)\footnote{http://opensource.org/licenses/bsd-license.php} license and thus is free for commercial or research use.
The library provides a wide range of implemented algorithms on image analyses, image data and matrix manipulation, basic and advanced image processing, object recognition and image feature extraction.

Most of Matchbox'es implementation is based on routines provided by OpenCV. 

\paragraph{Installing OpenCV}


\subsubsection{Python}

Python is a popular dynamic programming language. 
Python is used to model diverse use cases and workflows of document image quality assurance.

Modules required by Matchbox:

\begin{itemize}
	\item Numpy - Numerical Python
	\item args???
\end{itemize}


\paragraph{Installing Python}


\subsubsection{CMake}


The binaries can also be built manually using cmake, as long as the OpenCV development files can be found on the system.
To build the binaries, the following commands have to be issued in the \verb+pc-qa-matchbox+directory:

\paragraph{Installing CMake}



\subsubsection{C++ Compiler}


\paragraph{Linux}


\paragraph{Windows}



\subsection{Optional Packages}

\subsubsection{Intel Threading Building Blocks Library (TBB)}



\subsection{Installation on Linux}




\subsubsection{Ubuntu}
A debian package for the installation of Matchbox on various Linux distributions is available.
Matchbox has been developed using OpenCV 2.4, thus this library has to be present to install Matchbox. 
As of Ubuntu 12.04, there is no prebuilt package of the necessary OpenCV version in the Ubuntu repositories. 
The library can either be built by hand or a PPA can be added to the package sources (for more information see \url{https://help.launchpad.net/Packaging/PPA}). 
For example, while testing the debian package on a fresh Ubuntu installation, OpenCV was installed from \url{https://launchpad.net/~philip5/+archive/extra}, a PPA providing OpenCV 2.4. 

\subsubsection{other Linux distributions}
The debian package has not been tested yet for other debian-like distributions.
However, it should work in theory, as long as the OpenCV 2.4 packages are present.

\subsection{Installation on Windows}

\subsubsection{Compiling Matchbox using MS Visual Studio}

\subsubsection{Compiling Matchbox using MinGW}