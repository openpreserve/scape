\documentclass{hitec}
\usepackage[utf8]{inputenc}
\usepackage[T1]{fontenc}
%\usepackage{lmodern}
\usepackage{graphicx}
\usepackage{hyperref}

\title{Matchbox Usermanual (WIP)}


\begin{document}

\maketitle

\section{Matchbox}

Matchbox consists of simple commandline tools and some Python scripts.
The commandline tools have been deliberately kept simple so that they 
can be used in a variety of ways by connecting their inputs and outputs.

\subsection{extractfeatures}

{\ttfamily extractfeatures} gets various data from an image in a collection
and saves it to feature files.
It uses a SIFT feature detector to determine keypoints in an image, which are 
later used to compare the image with other images in the collection. Details
for the SIFT method can be found in~\cite{lowe}.

\begin{verbatim}
USAGE: 

extractfeatures  [--bow <filename>] [--precluster <int>] [--clahe <int>]
                 [--sdk <int>] [--numbins <int>] [-o <string>] ...  [-n
                 <string>] ...  [-l <int>] [-v] [-d <string>] [--]
                 [--version] [-h] <file>


Where: 

   --bow <filename>
     [BOWHistogram] Bag of Words file

   --precluster <int>
     [SIFTComparison] Number of descriptors to select in preclustering 
	 (0 = no preclustering)

   --clahe <int>
     [SIFTComparison] Value of adaptive contrast enhancement 
	 (1 = no enhancement)

   --sdk <int>
     [SIFTComparison] Number of Spatial Distinctive Keypoints 
	 (0 = no SDK)

   --numbins <int>
     [ImageHistogram] Number of histogram bins

   -o <string>,  --only <string>  (accepted multiple times)
     extract only specified features

   -n <string>,  --no <string>  (accepted multiple times)
     exclude feature from extraction

   -l <int>,  --level <int>
     Characterization Level (1-3)

   -v,  --verbose
     Provide additional debugging output

   -d <string>,  --dir <string>
     Output directory for feature files.

   --,  --ignore_rest
     Ignores the rest of the labeled arguments following this flag.

   --version
     Displays version information and exits.

   -h,  --help
     Displays usage information and exits.

   <file>
     (required)  image file to extract features from
\end{verbatim}

\subsection{train}

\texttt{train} goes through all the image descriptors found an creates the
actual ``Bag of Words''. 

\begin{verbatim}
USAGE: 

train  [-v] [-b <int>] [-p <int>] [-f <string>] -o <string> [--]
       [--version] [-h] <directories> ...


Where: 

   -v,  --verbose
     Provide additional debugging output

   -b <int>,  --bowsize <int>
     Size of the BoW Dictionary

   -p <int>,  --precluster <int>
     Number of descriptors to select in precluster-preprocessing 
	 (0 = no preclustering)

   -f <string>,  --filter <string>
     Filter files according to pattern

   -o <string>,  --output <string>
     (required)  Output file

   --,  --ignore_rest
     Ignores the rest of the labeled arguments following this flag.

   --version
     Displays version information and exits.

   -h,  --help
     Displays usage information and exits.

   <directories>  (accepted multiple times)
     (required)  Directory containing files with extracted features
\end{verbatim}

\subsection{compare}

\texttt{compare} carries out the actual comparison of the ``Bag of Words''
histograms and is also capable of calculating the structural similarity index
for spatial verification of possible duplicates.

\begin{verbatim}
USAGE: 

compare  [--bowmetric <str>] [--metric <str>] [-v] [-l <int>] [--]
         [--version] [-h] <file1> <file2>


Where: 

   --bowmetric <str>
     [BOWHistogram] Metric for histogram comparison (CV_COMP_CHISQR
     ,CV_COMP_CORREL,CV_COMP_INTERSECT,CV_COMP_BHATTACHARYYA)

   --metric <str>
     [ImageHistogram] Metric for histogram comparison (CV_COMP_CHISQR
     ,CV_COMP_CORREL,CV_COMP_INTERSECT,CV_COMP_BHATTACHARYYA)

   -v,  --verbose
     Provide additional debugging output

   -l <int>,  --level <int>
     Comparison Level (1-4)

   --,  --ignore_rest
     Ignores the rest of the labeled arguments following this flag.

   --version
     Displays version information and exits.

   -h,  --help
     Displays usage information and exits.

   <file1>
     (required)  file 1 that should be compared

   <file2>
     (required)  file 2 that should be compared
\end{verbatim}

\section{Workflows}

\subsection{Duplicate Detection - \ttfamily FindDuplicates.py}

FindDuplicates.py provides a way to search for duplicates in a collection of
image files using the approach explained in \cite{ait1}. A ``Bag of visual Words'' 
is calculated through clustering of image feature keypoints. For 
every image, ``Bag of Words'' histograms (``fingerprints'') are calculated. Images are
compared through histogram intersection and are ranked by a nearest neighbour algorithm. 
Possible duplicates are spatially verified by
calculation of structural similarity as described in \cite{ssim}.


Issuing \verb+python FindDuplicates.py+ without arguments prints a usage message.
\clearpage
\begin{verbatim}
usage: 

FindDuplicates.py [-h] [--threads THREADS] [--filter FILTER]
                  [--sdk SDK] [--nn NN] [--precluster PRECLUSTER]
                  [--clahe CLAHE] [--config CONFIG] [--featdir FEATDIR]
                  [--csv] [--bowsize BOWSIZE] [-v]
                  dir {all,extract,compare,train,bowhist,clean}
\end{verbatim}

\paragraph{\ttfamily-{}-threads} 

sets the number of threads to use to extract the features and to create the Bag
of Words. A number of 4 to 8 threads is reasonable, depending on the computer in use.

\paragraph{\ttfamily-{}-filter}

The filter argument that is passed to {\ttfamily train}. This determines which files in
FEATDIR are treated as feature files. The default is ``.SIFTComparison.feat.xml.gz''.

\paragraph{\ttfamily-{}-sdk}

The number of spatially distinctive keypoints {\ttfamily extractfeatures} will extract from
an image. By default a value of 2000 is used here.

\paragraph{\ttfamily-{}-nn NN}

This option sets the number of nearest neighbours that will be compared to the query
image. 

\paragraph{\ttfamily-{}-precluster}
The precluster argument that is passed to {\ttfamily train}.
Number of descriptors to select in precluster-preprocessing (0 = no preclustering).

\paragraph{\ttfamily-{}-clahe CLAHE}

This option enables the preprocessing of the images in a collection with CLAHE 
(contrast limited adaptive histogram equalization) to improve the sharpness
of the images. To be effective, a value greater than 1 (??) has to be supplied.

%\paragraph{\ttfamily-{}-config}

\paragraph{\ttfamily-{}-featdir FEATDIR}

With this option set, the tool looks for old feature files in FEATDIR and saves
newly extracted feature files to this directory.


\paragraph{\ttfamily-{}-csv}

Not implemented yet.

\paragraph{\ttfamily-{}-bowsize}

This option sets the number of ``visual words'' that will be
calculated for the Bag of Words.

The default value is 1000.

\paragraph{\ttfamily-v}

This enables more verbose output.

\paragraph{\ttfamily dir}

This option sets the folder where the collection is located.

\subsubsection*{Possible workflow steps}

\paragraph{\ttfamily all}

All workflow steps will be carried out. This includes extracting features,
calculating the Bag of Words, creating the Bag of Words histograms, calculating 
the histogram intersections of the images and calculating the SSIM for spatial verfication.

\paragraph{\ttfamily extract}

This extracts keypoints and feature descriptors from the images in the collection.
If a feature file already exists in the directory passed with the {\ttfamily -{}-featdir} option,
features are not extracted a second time.

\paragraph{\ttfamily train}

This step calculates the Bag of Words.

\paragraph{\ttfamily bowhist}

In this step, the Bag of Words histograms from the images in the collection are extracted.
If a Bag of Words histogram file already exists in the directory passed with the {\ttfamily -{}-featdir} option,
histograms are not extracted a second time.

\paragraph{\ttfamily compare}

In this step, the histogram intersection and spatial verification is done for every image.

\paragraph{\ttfamily clean}

This removes any feature files from the feature directory or the collections directory
depending on wether the {\ttfamily -{}-featdir} option was set or not.


\subsubsection*{Adding files to an existing collection}

After adding files to an existing collection, it is sufficient to execute
FindDuplicates.py with the right --featdir option twice, once with the option
extract and once with the option bowhist. The existing feature files will be
found and features/histograms will be extracted only for the newly added files.
This way, the computationally expensive step of creating the bag of words is
omitted. 

\subsection{Comparing Collections - CompareCollections.py}

The purpose of {\ttfamily CompareCollections.py} is to detect differences 
between two image collections that have been generated from the same source (book, newspaper, etc..).
For every image in the first collection, the tool tries to find a corresponding image in the second collection
in a similar manner as {\ttfamily FindDuplicates.py} does for images in a single collection.



\begin{verbatim}
usage: 
CompareCollections.py [-h] [--threads THREADS] [--filter FILTER]
                      [--sdk SDK] [--nn NN] [--precluster PRECLUSTER]
                      [--clahe CLAHE] [--config CONFIG]
                      [--featdir FEATDIR] [--csv] [--bowsize BOWSIZE]
                      [-v]
                      dir1 dir2
                      {all,extract,duplicates,references,clean}
\end{verbatim}

\paragraph{\ttfamily -{}-threads THREADS}

sets the number of threads to use to extract the features and to create the Bag
of Words. A number of 4 to 8 threads is reasonable, depending on the computer in use.

\paragraph{\ttfamily -{}-filter FILTER}

The filter argument that is passed to {\ttfamily train}. This determines which files in
FEATDIR are treated as feature files. The default is ``.SIFTComparison.feat.xml.gz''.

\paragraph{\ttfamily -{}-sdk SDK}

The number of spatially distinctive keypoints {\ttfamily extractfeatures} will extract from
an image. By default a value of 2000 is used here.

\paragraph{\ttfamily -{}-nn NN}

This option sets the number of nearest neighbours that will be compared to the query
image. 

\paragraph{\ttfamily -{}-precluster PRECLUSTER}

The precluster argument that is passed to {\ttfamily train}.
Number of descriptors to select in precluster-preprocessing (0 = no preclustering).

\paragraph{\ttfamily -{}-clahe CLAHE}

This option enables the preprocessing of the images in a collection with CLAHE 
(contrast limited adaptive histogram equalization) to improve the sharpness
of the images. To be effective, a value greater than 1 (??) has to be supplied.

%\paragraph{\ttfamily -{}-config CONFIG}

\paragraph{\ttfamily -{}-featdir FEATDIR}

With this option set, the tool looks for old feature files in FEATDIR and saves
newly extracted feature files to this directory.

\paragraph{\ttfamily -{}-csv}

Not implemented yet.

\paragraph{\ttfamily -{}-bowsize BOWSIZE}

This option sets the number of ``visual words'' that will be
calculated for the Bag of Words.

The default value is 1000.

\paragraph{\ttfamily -v}

This enables more verbose output.

\paragraph{\ttfamily dir1 dir2}

The two directories that contain the collections to be compared.

\subsubsection*{Possible workflow steps}


\appendix

\section{Troubleshooting}

\subsection{Old versions of Python - missing libraries}

The utility scripts in Matchbox were developed in Python 2.7. In older Python
versions some of the libraries that come with 2.7 are not installed out of the box. 
To use the utility scripts, these libraries have to be installed in one of Python's
library paths. They can be downloaded from \url{http://pypi.pyhton.org/}.

\subsection{New versions of Python - wrong syntax}

Attempts to use the utility scripts with Python 3.x will fail because
significant parts of the language have been changed from version 2.x to version
3.x. However, it is possible to install both versions in parallel on most
systems.

\subsection{Installation on Linux}

\subsubsection*{CMake}
The binaries can also be 
built manually using cmake, as long as the OpenCV development files can be found on the system.
To build the binaries, the following commands have to be issued in the \verb+pc-qa-matchbox+
directory:
\begin{verbatim}
cmake .
make
sudo make install
\end{verbatim}
If the compilation suceeds, the commands should be available at the commandline.

%TODO: describe prerequisites for compilation (gcc version? etc..)


\subsubsection*{Ubuntu}
A debian package for the installation of Matchbox on various Linux distributions
is available. Matchbox has been developed using OpenCV 2.4, thus this
library has to be present to install Matchbox. 
As of Ubuntu 12.04, there is no prebuilt package of the necessary OpenCV version
in the Ubuntu repositories. The library can either be built by hand or a PPA can be
added to the package sources (for more information see \url{https://help.launchpad.net/Packaging/PPA}). 
For example, while testing the debian package on a fresh Ubuntu installation, OpenCV was installed
from \url{https://launchpad.net/~philip5/+archive/extra}, a PPA providing OpenCV 2.4. 

\subsubsection*{other Linux distributions}
The debian package has not been tested yet for other debian-like distributions.
However, it should work in theory, as long as the OpenCV 2.4 packages are present.

\subsection{Compile Matchbox in Windows with CMake and MinGW or VisualStudio}

%TODO


\begin{thebibliography}{9}
\bibitem{ait1}
  Reinhold Huber-M\"ork, Alexander Schindler, Sven Schlarb, 
  \emph{Duplicate Detection for Quality Assurance of Document Image Collections}, 
  2012.
\bibitem{ssim}
	Z. Wang, A. C. Bovik, H. R. Sheikh and E. P. Simoncelli, 
	\emph{Image quality assessment: From error visibility to structural similarity}, 
	IEEE Transactions on Image Processing, vol. 13, no. 4, pp. 600-612, Apr. 2004.
\bibitem{lowe}
	Lowe, David G. (1999), 
	\emph{Object recognition from local scale-invariant features}, 
	Proceedings of the International Conference on Computer Vision. 2. pp. 1150-1157. 
\end{thebibliography}
\end{document}